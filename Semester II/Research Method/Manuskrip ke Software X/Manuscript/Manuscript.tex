\documentclass[preprint, 3p,
authoryear]{elsarticle} %review=doublespace preprint=single 5p=2 column
%%% Begin My package additions %%%%%%%%%%%%%%%%%%%

\usepackage[hyphens]{url}

  \journal{Software X} % Sets Journal name

\usepackage{lineno} % add

\usepackage{graphicx}
%%%%%%%%%%%%%%%% end my additions to header

\usepackage[T1]{fontenc}
\usepackage{lmodern}
\usepackage{amssymb,amsmath}
\usepackage{ifxetex,ifluatex}
\usepackage{fixltx2e} % provides \textsubscript
% use upquote if available, for straight quotes in verbatim environments
\IfFileExists{upquote.sty}{\usepackage{upquote}}{}
\ifnum 0\ifxetex 1\fi\ifluatex 1\fi=0 % if pdftex
  \usepackage[utf8]{inputenc}
\else % if luatex or xelatex
  \usepackage{fontspec}
  \ifxetex
    \usepackage{xltxtra,xunicode}
  \fi
  \defaultfontfeatures{Mapping=tex-text,Scale=MatchLowercase}
  \newcommand{\euro}{€}
\fi
% use microtype if available
\IfFileExists{microtype.sty}{\usepackage{microtype}}{}
\usepackage[]{natbib}
\bibliographystyle{plainnat}

\ifxetex
  \usepackage[setpagesize=false, % page size defined by xetex
              unicode=false, % unicode breaks when used with xetex
              xetex]{hyperref}
\else
  \usepackage[unicode=true]{hyperref}
\fi
\hypersetup{breaklinks=true,
            bookmarks=true,
            pdfauthor={},
            pdftitle={An R-base program to build e-commerce product discount portfolio using spiral dynamic optimization algorithm},
            colorlinks=false,
            urlcolor=blue,
            linkcolor=magenta,
            pdfborder={0 0 0}}

\setcounter{secnumdepth}{5}
% Pandoc toggle for numbering sections (defaults to be off)


% tightlist command for lists without linebreak
\providecommand{\tightlist}{%
  \setlength{\itemsep}{0pt}\setlength{\parskip}{0pt}}

% From pandoc table feature
\usepackage{longtable,booktabs,array}
\usepackage{calc} % for calculating minipage widths
% Correct order of tables after \paragraph or \subparagraph
\usepackage{etoolbox}
\makeatletter
\patchcmd\longtable{\par}{\if@noskipsec\mbox{}\fi\par}{}{}
\makeatother
% Allow footnotes in longtable head/foot
\IfFileExists{footnotehyper.sty}{\usepackage{footnotehyper}}{\usepackage{footnote}}
\makesavenoteenv{longtable}





\begin{document}


\begin{frontmatter}

  \title{An R-base program to build e-commerce product discount
portfolio using spiral dynamic optimization algorithm}
    \author[lala]{Mohammad Rizka Fadhli%
  \corref{cor1}%
  }
   \ead{20921004@mahasiswa.itb.ac.id} 
    \author[lili]{Saladin Uttunggadewa%
  %
  }
  
    \author[lulu]{Rieske Hadianti%
  %
  }
  
    \author[lili]{Sri Redjeki Pudjaprasetya%
  %
  }
  
      \affiliation[lala]{Magister of Computational Sciences Program,
Faculty of Mathematics and Natural Sciences, Institut Teknologi Bandung}
    \affiliation[lili]{Faculty of Mathematics and Natural Sciences,
Institut Teknologi Bandung}
    \affiliation[lulu]{Center for Mathematical Modeling and Simulation,
Institut Teknologi Bandung}
    \cortext[cor1]{Corresponding author}
  
  \begin{abstract}
  The development of the digital economy market in Indonesia is growing
  every year. As a result, marketplaces have to compete with each other
  to get consumers. One of the marketplace startup in Indonesia has
  strategy to provide additional discounts on their listed products.
  Product discount portfolio problem is a Binary Linear Programming
  (BLP) problem with a decision variable in the form of a binary number
  that states whether a product is eligible for a discount or not. With
  many decision variables involved, we developed metaheuristic
  approaches Spiral Dynamic Optimization Algorithm (SDOA) to solve the
  optimization problem. The SDOA solution is proven to be more optimal
  (generates higher revenur) than the existing startup solution by
  average \(9.8 \%\).
  \end{abstract}
    \begin{keyword}
    binary linear programming \sep spiral dynamic optimization
algorithm \sep R programming \sep 
    e-commerce
  \end{keyword}
  
 \end{frontmatter}

\hypertarget{introduction}{%
\section{INTRODUCTION}\label{introduction}}

Currently, 30 million Indonesians are used to buying and selling online,
creating a market of 8 trillion rupiahs. This market can continue to
grow to 40 trillion in the next five years. Online transactions are
divided into social commerce (trading via social media platforms) and
e-commerce (trading via marketplace platforms). By 2022, the revenue
projection from the e-commerce market will exceed US\$62 million. It was
highly contributed by the increasing number of MSMEs that listed their
products in e-commerce. The total number of MSMEs that market their
products in e-commerce is currently 14.5 million MSMEs. The number has
not reached half of the target set in 2023, which is 30 million MSMEs.

One of the top marketplace in Indonesia use a product discount strategy
to win the competition. Each listed product on the marketplace has an
initial price defined by the seller. Marketplace can intervene in its
price indirectly by providing discounts, so consumers pay lower than the
initial price. A study at an online retailer in China shows an influence
between product discounts and consumer purchase behavior such as
purchase incident, purchase quantity, and spending. Especially in
specific discount percentage ranges \citep{cny}. The pricing and
discount aim to attract consumers to buy products in the marketplace for
a certain period. One of the mathematical models that can be used to
determine prices is the Price Elasticity Model, a causality model
between price and demand. A study comparing the price elasticity model
with linear, polynomial, and exponential bases found that the linear
model is the most commonly used, and it is easier to interpret
\citep{linear_reg}.

The problem is determining which products need to be given an additional
discount. This problem is one of the binary linear programming (BLP)
forms: choosing the right product to maximize sales with a constraint
limited discount budget in a certain period. Binary programming is a
form of an optimization method in which the variables involved are
binary numbers. The main characteristic of BLP is the matching process
between indexes \citep{lieberman}. One of the studies related to BLP is
the matching process to determine the placement of phasor measuring
units in a power system \citep{suresh}. Another study used BLP to
optimize the selection of 100 marketing channels and activities against
millions of customers \citep{ieee}. The goal is to optimize the
marketing messages delivery so that sales increase. Business-related and
consumer-centric optimization problems have not been widely published in
journals.

At least we can use two methods to solve BLP: the exact method and the
meta-heuristic method. Meta-heuristic methods are developing rapidly
that several algorithms can quickly solve optimization problems. The
meta-heuristic algorithm such as SDOA can solve BLP in a small number of
decision variables involved. SDOA is an algorithm inspired by spiral
motion in natural events. The rotation matrix is an algebraic principle
closely related to SDOA \citep{tamura}. SDOA is proven to be used to
solve BLP problems by modifying the existing objective function and
constraints \citep{kun}. The basic idea is to turn a constrained problem
into an unconstrained problem by creating a penalty function.

In another study, researchers made an adaptive linear SDOA by changing
the value of the contraction constant into a specific function that
depends on the function value of each candidate solution \citep{sdoa2}.
The latest study in 2022 conducted a thorough review, including making
several improvements to SDOA \citep{sdoa3}, such as:

\begin{itemize}
\tightlist
\item
  Changing the value of the contraction constant,
\item
  Utilizing alternative functions (linear, quadratic, fuzzy, and
  exponential), and
\item
  Crossing between SDOA and PSO into PSO with a spiral movement.
\end{itemize}

Currently, the marketplace determines which product to give a discount
by trial and error from several possible combinations - (this method is
similar to Monte Carlo simulation but on a simpler computational scale).
In this research, we develop a computational model based on SDOA to
solve the discount product portfolio optimization problem. Then we will
compare the solution with the existing product portfolio.

\hypertarget{method}{%
\section{METHOD}\label{method}}

The marketplace has created the following portfolio:

\begin{longtable}[]{@{}lrr@{}}
\caption{Sample Portfolio Data}\tabularnewline
\toprule()
product\_code & budget & expected\_revenue \\
\midrule()
\endfirsthead
\toprule()
product\_code & budget & expected\_revenue \\
\midrule()
\endhead
6000094-0002 & 240.0 & 112.800 \\
6000100-0003 & 70350.0 & 78289.500 \\
6000301-0004 & 15300.0 & 7191.000 \\
6000307-0005 & 2700.0 & 2079.000 \\
6000348-0007 & 50460.0 & 55036.200 \\
6000378-0010 & 1425.0 & 1097.250 \\
6000514-0014 & 28774200.0 & 15355284.000 \\
6000134-0026 & 3072.0 & 1443.840 \\
6000149-0029 & 2052.0 & 964.440 \\
6000245-0036 & 5670.0 & 1449.900 \\
6000014-0038 & 65016.0 & 41897.520 \\
6000027-0040 & 5625.0 & 4331.250 \\
6000030-0041 & 21168.0 & 22548.960 \\
6000042-0044 & 6747.3 & 1725.381 \\
6000062-0048 & 6349.5 & 2424.015 \\
\bottomrule()
\end{longtable}

Each portfolio contains 100 product candidates. The marketplace has the
following problem: choosing the right product from the portfolio to
generate the maximal revenue with a budget limit of 5 million rupiahs.

The problem can be written as follows:

\begin{align}
\begin{matrix} 
\text{find }x_i \in [0,1] \space \text{so that} \space \max{r_i x_i}, \space i \in \{1,2,..,100\} \\
\text{with constraint} \space \sum_{i = 1}^{100}x_i b_i \leq 5000000
\end{matrix} 
\end{align}

whereas \(r_i\) is expected revenue if the product \(i\) is given a
discount and \(b_i\) is the amount of budget that needs to be spent to
give a discount on a product \(i\).

To solve using SDOA, we need to convert the constrained problem to an
unconstrained problem. The general form is as follows: for any MILP or
MIP:

\begin{align}
\begin{matrix} 
\min_{x \in \mathbb{R}} f(x) \\
\text{subject to } \space h(x) \leq 0 \\
x = (x_1,x_2,..,x_n)^T \in \mathbb{R}
\end{matrix}
\end{align}

The general unconstrained form is as follows:

\begin{align}
F(x,\beta) = f(x) + \beta \max(h(x),0)^2
\end{align}

where \(\beta\) is penalty constant (defined by large number). Finding
\(x\) so that \(F(x,\beta)\) is minimum.

This is general form of spiral dynamic optimization algorithm:

\begin{verbatim}
INPUT
  N number of candidate solutions
  theta rotation degree
    A(theta) is rotation matrix
  r contraction constant
  iter_max iteration limit
PROCESS:
  Step 1: Generate N random candidate solutions: xi in feasible region. 
  Step 2: set k = 0.
  Step 3: Evaluate all F(xi). Find x0 which makes F(x0) the minimum. Set x0 as rotation center.
  Step 4: Update xi(k+1) = x0 + r (A(theta) (xi(k+1) - x0)).
  Step 5: Update k = k + 1.
  Step 6: Repeat step 3 - 5 until k reach iter_max.
OUTPUT:
  The last x0 is the optimal solution which makes F(X0) the minimum.
\end{verbatim}

One of the crucial processes in SDOA is to rotate candidate solutions
using a rotation matrix. In this case, one candidate solution is a
vector of size 1 \(\times\) 100 while the rotation matrix is of size 100
\(\times\) 100. This process is quite computationally burdensome,
especially since the number of candidate solutions used is relatively
large. Therefore, we need to modify this rotation process by using
matrix multiplication partitioning in linear algebra.

For instance: vector b is 1 \(\times\) n, and matrix A is n \(\times\)
n.~Both can be divided into two, namely:

\begin{itemize}
\tightlist
\item
  Vector b1 is 1 \(\times \frac{n}{2}\) and vector b2 is 1
  \(\times \frac{n}{2}\). Vector b1 contains the 1st to
  \(\frac{n}{2}\)-th elements of vector b. Vector b2 contains elements
  \(\frac{n}{2} + 1\)-th to \(n\)-th of vector b.
\item
  Matrix A is divided into 4 for the same size matrix
  \(\frac{n}{2} \times \frac{n}{2}\).
\end{itemize}

\begin{align}
A = \begin{pmatrix}
a_{11} & a_{12} & .. & a_{1n} \\
a_{21} & a_{22} & .. & a_{2n} \\
a_{31} & a_{32} & .. & a_{3n} \\
.. & .. & .. \\
a_{n1} & a_{n2} & .. & a_{nn} \\
\end{pmatrix}
\end{align}

\begin{itemize}
\tightlist
\item
  Matrix A11 is the top-left element of matrix A.
\item
  Matrix A12 is the top-right element of matrix A.
\item
  Matrix A21 is the lower-left element of matrix A.
\item
  Matrix A22 is the lower-right element of matrix A.
\end{itemize}

\begin{align}
r = A b
\end{align}

So we get the result vector \(r\) with 1 \(\times\) n elements:

\begin{enumerate}
\def\labelenumi{\arabic{enumi}.}
\tightlist
\item
  Elements 1 to \(\frac{n}{2}\) = b1 * A11 + b2 * A12
\item
  Elements \(\frac{n}{2} + 1\) to \(n\) = b1 * A21 + b2 * A22
\end{enumerate}

The algorithm for solving optimization problems using modified SDOA was
made in R programming language. As a form of evaluation of the
algorithm, we compared the SDOA solution with:

\begin{enumerate}
\def\labelenumi{\arabic{enumi}.}
\tightlist
\item
  The existing solution from the marketplace.
\item
  The exact solution uses the simplex method created using the R
  programming language with \texttt{library(ompr)} \citep{ompr}.
\end{enumerate}

This comparison uses portfolio data from as many as ten portfolios with
100 products per portfolio each. Two factors that will be compared are
the total budget issued and the total expected revenue.

\hypertarget{results-and-discussion}{%
\section{RESULTS AND DISCUSSION}\label{results-and-discussion}}

SDOA is created using the following parameters:

\begin{enumerate}
\def\labelenumi{\arabic{enumi}.}
\tightlist
\item
  \emph{N} candidate solutions are made up of 200 candidates in the form
  of a binary vector measuring 1 \(\times\) 100.
\item
  \(\theta\) is \(\frac{50}{2 \pi}\).
\item
  Contraction constant of \(0.85\).
\item
  The iteration limit is \(70\).
\end{enumerate}

The following are some of the results of SDOA solutions in ten
portfolios and their comparison to existing solutions and exact
solutions.

\hypertarget{budget-comparison}{%
\subsection{Budget Comparison}\label{budget-comparison}}

First, we compare the budget from the exact method
(\texttt{exact\_budget} using simplex) with the budget using the SDOA
and the existing budget from the marketplace.

\begin{longtable}[]{@{}rrrr@{}}
\caption{Budget Fullfilment: What percentage of the budget is spent on
the solution?}\tabularnewline
\toprule()
\#portfolio & exact\_budget & existing\_budget & spiral\_budget \\
\midrule()
\endfirsthead
\toprule()
\#portfolio & exact\_budget & existing\_budget & spiral\_budget \\
\midrule()
\endhead
1 & 100.00 & 93.96 & 99.98 \\
2 & 99.99 & 92.19 & 99.99 \\
3 & 100.00 & 89.16 & 99.99 \\
4 & 100.00 & 98.62 & 99.99 \\
5 & 100.00 & 95.91 & 99.96 \\
6 & 100.00 & 97.65 & 99.74 \\
7 & 100.00 & 89.16 & 99.51 \\
8 & 100.00 & 97.01 & 99.80 \\
9 & 96.85 & 72.53 & 95.97 \\
10 & 100.00 & 80.77 & 100.00 \\
\bottomrule()
\end{longtable}

The total discounted budget per portfolio is five million rupiahs. If we
look at the table above, each method produces a solution that does not
violate the total budget limit. However, the exact solution always
consumes all of the total budget, while the SDOA solution still has a
smaller budget remaining than the existing solution. These findings
found that the SDOA solution was better in using the total budget than
the existing solution.

\hypertarget{total-expected-revenue-comparison}{%
\subsection{Total Expected Revenue
Comparison}\label{total-expected-revenue-comparison}}

\begin{longtable}[]{@{}rlll@{}}
\caption{Total Expected Revenue: What is the percentage of revenue
achieved compared to the revenue of the exact solution?}\tabularnewline
\toprule()
\#portfolio & eksak\_revenue & existing\_revenue & spiral\_revenue \\
\midrule()
\endfirsthead
\toprule()
\#portfolio & eksak\_revenue & existing\_revenue & spiral\_revenue \\
\midrule()
\endhead
1 & 4.778 mil IDR & 81.67\% & 90.19\% \\
2 & 7.194 mil IDR & 74.56\% & 80.52\% \\
3 & 7.428 mil IDR & 34.93\% & 50.24\% \\
4 & 5.568 mil IDR & 88.26\% & 92.53\% \\
5 & 2.606 mil IDR & 81.92\% & 90.65\% \\
6 & 6.203 mil IDR & 94.74\% & 94.57\% \\
7 & 7.4 mil IDR & 74.04\% & 88.4\% \\
8 & 4.146 mil IDR & 91.13\% & 99.41\% \\
9 & 5.097 mil IDR & 82.27\% & 99.34\% \\
10 & 5.525 mil IDR & 80.41\% & 97.02\% \\
\bottomrule()
\end{longtable}

The following comparison shows how much total expected revenue can be
achieved. If the revenue from the exact solution is used as a benchmark
value (the highest revenue that can be achieved from each portfolio),
the percentage in the table is calculated by dividing the revenue from
the existing solution or SDOA by the revenue from the exact solution.
The revenue generated by the SDOA solution is 9.8\% higher (on average)
than the existing solution. However, the SDOA solution revenue is still
11.7\% lower (on average) than the exact solution.

\hypertarget{similarity-between-exact-and-sdoa-solution}{%
\subsection{Similarity Between Exact and SDOA
Solution}\label{similarity-between-exact-and-sdoa-solution}}

The exact solution is the most optimal, while SDOA has not been able to
achieve that optimality. What do the two have in common? Suppose we
count how many products are discounted or not discounted in both
portfolios (exact and SDOA); here are the results:

\begin{longtable}[]{@{}rrr@{}}
\caption{Similarity Product Portfolio}\tabularnewline
\toprule()
\#portfolio & similar\_product & dissimilar\_product \\
\midrule()
\endfirsthead
\toprule()
\#portfolio & similar\_product & dissimilar\_product \\
\midrule()
\endhead
1 & 63 & 37 \\
2 & 36 & 64 \\
3 & 55 & 45 \\
4 & 59 & 41 \\
5 & 74 & 26 \\
6 & 48 & 52 \\
7 & 63 & 37 \\
8 & 85 & 15 \\
9 & 91 & 9 \\
10 & 68 & 32 \\
\bottomrule()
\end{longtable}

On average, the similarity between the two portfolios is 64.2\%. Now we
see what proportion of products are given and not discounted on the
exact method and SDOA.

\begin{longtable}[]{@{}
  >{\raggedleft\arraybackslash}p{(\columnwidth - 4\tabcolsep) * \real{0.1549}}
  >{\raggedright\arraybackslash}p{(\columnwidth - 4\tabcolsep) * \real{0.4225}}
  >{\raggedright\arraybackslash}p{(\columnwidth - 4\tabcolsep) * \real{0.4225}}@{}}
\caption{Product Proportion: How many products are given and not
discounted?}\tabularnewline
\toprule()
\begin{minipage}[b]{\linewidth}\raggedleft
\#portfolio
\end{minipage} & \begin{minipage}[b]{\linewidth}\raggedright
portofolio\_eksak
\end{minipage} & \begin{minipage}[b]{\linewidth}\raggedright
portofolio\_spiral
\end{minipage} \\
\midrule()
\endfirsthead
\toprule()
\begin{minipage}[b]{\linewidth}\raggedleft
\#portfolio
\end{minipage} & \begin{minipage}[b]{\linewidth}\raggedright
portofolio\_eksak
\end{minipage} & \begin{minipage}[b]{\linewidth}\raggedright
portofolio\_spiral
\end{minipage} \\
\midrule()
\endhead
1 & Discount: 69; No discount: 31 & Discount: 62; No Discount: 38 \\
2 & Discount: 14; No discount: 86 & Discount: 72; No Discount: 28 \\
3 & Discount: 33; No discount: 67 & Discount: 56; No Discount: 44 \\
4 & Discount: 54; No discount: 46 & Discount: 69; No Discount: 31 \\
5 & Discount: 87; No discount: 13 & Discount: 73; No Discount: 27 \\
6 & Discount: 42; No discount: 58 & Discount: 72; No Discount: 28 \\
7 & Discount: 62; No discount: 38 & Discount: 83; No Discount: 17 \\
8 & Discount: 86; No discount: 14 & Discount: 89; No Discount: 11 \\
9 & Discount: 99; No discount: 1 & Discount: 90; No Discount: 10 \\
10 & Discount: 82; No discount: 18 & Discount: 74; No Discount: 26 \\
\bottomrule()
\end{longtable}

Although there is a 64.2\% similarity, the differences in the range of
selected products in the two portfolios are apparent. The SDOA solution
can still not explore the possibility of the optimal solution like an
exact solution in the feasible area.

\hypertarget{conclusion}{%
\section{CONCLUSION}\label{conclusion}}

SDOA is proven to provide better solutions than existing solutions,
namely: spending better budgets and generating more revenue. However, to
achieve an optimal solution, such as an exact solution, further
modification is needed to make SDOA explore more potential candidates in
a feasible area. Some alternatives to consider are:

\begin{enumerate}
\def\labelenumi{\arabic{enumi}.}
\tightlist
\item
  Using a pseudo-random algorithm in generating candidate solutions at
  an early stage.
\item
  Multiply N candidate solutions made.
\end{enumerate}

\hypertarget{declarations}{%
\section{DECLARATIONS}\label{declarations}}

\hypertarget{study-limitations}{%
\subsection{Study Limitations}\label{study-limitations}}

In this study, the portfolio used is derived from the calculation of the
price elasticity study previously conducted by the marketplace data
analytic team.

\begin{itemize}
\tightlist
\item
  The basic assumption of this portfolio is that changes in sales only
  come from changes in product prices. The interaction between products
  in the same category (competitor effect) is not considered in this
  study.
\item
  The budget and expected revenue per product are assumed to be fixed
  values (not dynamic or probabilistic).
\item
  The total revenue calculation achieved from the solution of each
  method in the previous section is derived from the value in the
  portfolio.
\end{itemize}

\hypertarget{acknowledgements}{%
\subsection{Acknowledgements}\label{acknowledgements}}

The author would like to thank the marketplace data analytic team for
entrusting its product portfolio data to be used as a case study in
developing an SDOA-based optimization algorithm.

\hypertarget{declaration-of-competing-interest}{%
\subsection{Declaration of Competing
Interest}\label{declaration-of-competing-interest}}

The author stated that there was no conflict of interest during this
research.

\renewcommand\refname{REFERENCES}
\bibliography{mybibfile.bib}


\end{document}
