% Options for packages loaded elsewhere
\PassOptionsToPackage{unicode}{hyperref}
\PassOptionsToPackage{hyphens}{url}
%
\documentclass[
]{article}
\usepackage{amsmath,amssymb}
\usepackage{lmodern}
\usepackage{iftex}
\ifPDFTeX
  \usepackage[T1]{fontenc}
  \usepackage[utf8]{inputenc}
  \usepackage{textcomp} % provide euro and other symbols
\else % if luatex or xetex
  \usepackage{unicode-math}
  \defaultfontfeatures{Scale=MatchLowercase}
  \defaultfontfeatures[\rmfamily]{Ligatures=TeX,Scale=1}
\fi
% Use upquote if available, for straight quotes in verbatim environments
\IfFileExists{upquote.sty}{\usepackage{upquote}}{}
\IfFileExists{microtype.sty}{% use microtype if available
  \usepackage[]{microtype}
  \UseMicrotypeSet[protrusion]{basicmath} % disable protrusion for tt fonts
}{}
\makeatletter
\@ifundefined{KOMAClassName}{% if non-KOMA class
  \IfFileExists{parskip.sty}{%
    \usepackage{parskip}
  }{% else
    \setlength{\parindent}{0pt}
    \setlength{\parskip}{6pt plus 2pt minus 1pt}}
}{% if KOMA class
  \KOMAoptions{parskip=half}}
\makeatother
\usepackage{xcolor}
\IfFileExists{xurl.sty}{\usepackage{xurl}}{} % add URL line breaks if available
\IfFileExists{bookmark.sty}{\usepackage{bookmark}}{\usepackage{hyperref}}
\hypersetup{
  pdftitle={Untitled},
  pdfauthor={Mohammad Rizka Fadhli},
  hidelinks,
  pdfcreator={LaTeX via pandoc}}
\urlstyle{same} % disable monospaced font for URLs
\usepackage[margin=1in]{geometry}
\usepackage{color}
\usepackage{fancyvrb}
\newcommand{\VerbBar}{|}
\newcommand{\VERB}{\Verb[commandchars=\\\{\}]}
\DefineVerbatimEnvironment{Highlighting}{Verbatim}{commandchars=\\\{\}}
% Add ',fontsize=\small' for more characters per line
\usepackage{framed}
\definecolor{shadecolor}{RGB}{248,248,248}
\newenvironment{Shaded}{\begin{snugshade}}{\end{snugshade}}
\newcommand{\AlertTok}[1]{\textcolor[rgb]{0.94,0.16,0.16}{#1}}
\newcommand{\AnnotationTok}[1]{\textcolor[rgb]{0.56,0.35,0.01}{\textbf{\textit{#1}}}}
\newcommand{\AttributeTok}[1]{\textcolor[rgb]{0.77,0.63,0.00}{#1}}
\newcommand{\BaseNTok}[1]{\textcolor[rgb]{0.00,0.00,0.81}{#1}}
\newcommand{\BuiltInTok}[1]{#1}
\newcommand{\CharTok}[1]{\textcolor[rgb]{0.31,0.60,0.02}{#1}}
\newcommand{\CommentTok}[1]{\textcolor[rgb]{0.56,0.35,0.01}{\textit{#1}}}
\newcommand{\CommentVarTok}[1]{\textcolor[rgb]{0.56,0.35,0.01}{\textbf{\textit{#1}}}}
\newcommand{\ConstantTok}[1]{\textcolor[rgb]{0.00,0.00,0.00}{#1}}
\newcommand{\ControlFlowTok}[1]{\textcolor[rgb]{0.13,0.29,0.53}{\textbf{#1}}}
\newcommand{\DataTypeTok}[1]{\textcolor[rgb]{0.13,0.29,0.53}{#1}}
\newcommand{\DecValTok}[1]{\textcolor[rgb]{0.00,0.00,0.81}{#1}}
\newcommand{\DocumentationTok}[1]{\textcolor[rgb]{0.56,0.35,0.01}{\textbf{\textit{#1}}}}
\newcommand{\ErrorTok}[1]{\textcolor[rgb]{0.64,0.00,0.00}{\textbf{#1}}}
\newcommand{\ExtensionTok}[1]{#1}
\newcommand{\FloatTok}[1]{\textcolor[rgb]{0.00,0.00,0.81}{#1}}
\newcommand{\FunctionTok}[1]{\textcolor[rgb]{0.00,0.00,0.00}{#1}}
\newcommand{\ImportTok}[1]{#1}
\newcommand{\InformationTok}[1]{\textcolor[rgb]{0.56,0.35,0.01}{\textbf{\textit{#1}}}}
\newcommand{\KeywordTok}[1]{\textcolor[rgb]{0.13,0.29,0.53}{\textbf{#1}}}
\newcommand{\NormalTok}[1]{#1}
\newcommand{\OperatorTok}[1]{\textcolor[rgb]{0.81,0.36,0.00}{\textbf{#1}}}
\newcommand{\OtherTok}[1]{\textcolor[rgb]{0.56,0.35,0.01}{#1}}
\newcommand{\PreprocessorTok}[1]{\textcolor[rgb]{0.56,0.35,0.01}{\textit{#1}}}
\newcommand{\RegionMarkerTok}[1]{#1}
\newcommand{\SpecialCharTok}[1]{\textcolor[rgb]{0.00,0.00,0.00}{#1}}
\newcommand{\SpecialStringTok}[1]{\textcolor[rgb]{0.31,0.60,0.02}{#1}}
\newcommand{\StringTok}[1]{\textcolor[rgb]{0.31,0.60,0.02}{#1}}
\newcommand{\VariableTok}[1]{\textcolor[rgb]{0.00,0.00,0.00}{#1}}
\newcommand{\VerbatimStringTok}[1]{\textcolor[rgb]{0.31,0.60,0.02}{#1}}
\newcommand{\WarningTok}[1]{\textcolor[rgb]{0.56,0.35,0.01}{\textbf{\textit{#1}}}}
\usepackage{longtable,booktabs,array}
\usepackage{calc} % for calculating minipage widths
% Correct order of tables after \paragraph or \subparagraph
\usepackage{etoolbox}
\makeatletter
\patchcmd\longtable{\par}{\if@noskipsec\mbox{}\fi\par}{}{}
\makeatother
% Allow footnotes in longtable head/foot
\IfFileExists{footnotehyper.sty}{\usepackage{footnotehyper}}{\usepackage{footnote}}
\makesavenoteenv{longtable}
\usepackage{graphicx}
\makeatletter
\def\maxwidth{\ifdim\Gin@nat@width>\linewidth\linewidth\else\Gin@nat@width\fi}
\def\maxheight{\ifdim\Gin@nat@height>\textheight\textheight\else\Gin@nat@height\fi}
\makeatother
% Scale images if necessary, so that they will not overflow the page
% margins by default, and it is still possible to overwrite the defaults
% using explicit options in \includegraphics[width, height, ...]{}
\setkeys{Gin}{width=\maxwidth,height=\maxheight,keepaspectratio}
% Set default figure placement to htbp
\makeatletter
\def\fps@figure{htbp}
\makeatother
\setlength{\emergencystretch}{3em} % prevent overfull lines
\providecommand{\tightlist}{%
  \setlength{\itemsep}{0pt}\setlength{\parskip}{0pt}}
\setcounter{secnumdepth}{-\maxdimen} % remove section numbering
\ifLuaTeX
  \usepackage{selnolig}  % disable illegal ligatures
\fi

\title{Untitled}
\author{Mohammad Rizka Fadhli}
\date{5/19/2022}

\begin{document}
\maketitle

\newpage

\hypertarget{soal-dan-pembahasan}{%
\section{SOAL DAN PEMBAHASAN}\label{soal-dan-pembahasan}}

\hypertarget{soal-i}{%
\subsection{Soal I}\label{soal-i}}

Diberikan 10 buah titik data sebagai berikut:

\begin{longtable}[]{@{}lrr@{}}
\caption{Data Soal I}\tabularnewline
\toprule
titik & x & y \\
\midrule
\endfirsthead
\toprule
titik & x & y \\
\midrule
\endhead
p1 & 4.0 & 5.2 \\
p2 & 2.1 & 3.9 \\
p3 & 3.4 & 3.1 \\
p4 & 2.7 & 2.0 \\
p5 & 0.8 & 4.1 \\
p6 & 4.6 & 2.9 \\
p7 & 4.3 & 1.2 \\
p8 & 2.2 & 1.0 \\
p9 & 4.1 & 4.1 \\
p10 & 1.5 & 3.0 \\
\bottomrule
\end{longtable}

\begin{itemize}
\tightlist
\item
  Lakukan klasterisasi dari data tersebut dengan menggunakan algoritma
  \emph{k-means} dengan jumlah partisi \(K=2\) sebanyak 10 kali.
\item
  Tentukan sentroid awal (secara \emph{random}) yang berbeda setiap
  melakukan klasterisasi.
\item
  \emph{Stopping criteria} untuk klasterisasi bisa ditentukan sendiri
  (tidak harus sampai tidak ada perubahan sentroid)
\end{itemize}

\hypertarget{pertanyaan}{%
\subsubsection{Pertanyaan}\label{pertanyaan}}

\begin{enumerate}
\def\labelenumi{\arabic{enumi}.}
\tightlist
\item
  Tuliskan hasil akhir kluster yang didapat untuk setiap klasterisasi!
\item
  Hitung nilai \emph{average} \textbf{\emph{SSE}} untuk masing-masing
  hasil klusterisasi!
\item
  Hitung nilai \emph{average} \textbf{\emph{Sillhouette Coefficient}}
  untuk masing-masing hasil klusterisasi!
\item
  Dari hasil \textbf{\emph{SSE}} dan \textbf{\emph{Sillhouette
  Coefficient}}, menurut Anda, hasil klasterisasi mana yang memberikan
  hasil terbaik? Berikan alasannya!
\item
  Apakah algoritma \emph{K-means} sudah memberikan hasil yang baik? Apa
  yang dapat dilakukan agar hasil klasterisasi lebih baik?
\end{enumerate}

\hypertarget{pembahasan}{%
\subsubsection{Pembahasan}\label{pembahasan}}

Untuk melakukan \emph{k-means clustering} ini, saya akan membuat
algoritma sendiri dengan menggunakan 2 titik \emph{random} dan akan
dilakukan sebanyak 10 kali.

\begin{Shaded}
\begin{Highlighting}[]
\CommentTok{\# program untuk membuat titik sentroid secara random}
\NormalTok{random\_titik }\OtherTok{=} \ControlFlowTok{function}\NormalTok{()\{}
  \FunctionTok{list}\NormalTok{(}
    \AttributeTok{sentroid\_1 =} \FunctionTok{runif}\NormalTok{(}\DecValTok{2}\NormalTok{,}\DecValTok{0}\NormalTok{,}\DecValTok{6}\NormalTok{),}
    \AttributeTok{sentroid\_2 =} \FunctionTok{runif}\NormalTok{(}\DecValTok{2}\NormalTok{,}\DecValTok{0}\NormalTok{,}\DecValTok{6}\NormalTok{)}
\NormalTok{  )}
\NormalTok{\}}

\CommentTok{\# program untuk menghitung jarak}
\NormalTok{jarak }\OtherTok{=} \ControlFlowTok{function}\NormalTok{(x1,x2)\{}
\NormalTok{  sb\_1 }\OtherTok{=}\NormalTok{ (x1[}\DecValTok{1}\NormalTok{] }\SpecialCharTok{{-}}\NormalTok{ x2[}\DecValTok{1}\NormalTok{])}\SpecialCharTok{\^{}}\DecValTok{2}
\NormalTok{  sb\_2 }\OtherTok{=}\NormalTok{ (x1[}\DecValTok{2}\NormalTok{] }\SpecialCharTok{{-}}\NormalTok{ x2[}\DecValTok{2}\NormalTok{])}\SpecialCharTok{\^{}}\DecValTok{2}
  \FunctionTok{sqrt}\NormalTok{(sb\_1 }\SpecialCharTok{+}\NormalTok{ sb\_2)}
\NormalTok{\}}
\end{Highlighting}
\end{Shaded}

\begin{Shaded}
\begin{Highlighting}[]
\CommentTok{\# iterasi pertama}
\NormalTok{random }\OtherTok{=} \FunctionTok{random\_titik}\NormalTok{()}
\NormalTok{sentroid\_1 }\OtherTok{=}\NormalTok{ random}\SpecialCharTok{$}\NormalTok{sentroid\_1}
\NormalTok{sentroid\_2 }\OtherTok{=}\NormalTok{ random}\SpecialCharTok{$}\NormalTok{sentroid\_2}

\NormalTok{df}\SpecialCharTok{$}\NormalTok{jarak\_sentroid1 }\OtherTok{=} \ConstantTok{NA}
\NormalTok{df}\SpecialCharTok{$}\NormalTok{jarak\_sentroid2 }\OtherTok{=} \ConstantTok{NA}

\ControlFlowTok{for}\NormalTok{(i }\ControlFlowTok{in} \DecValTok{1}\SpecialCharTok{:}\FunctionTok{nrow}\NormalTok{(df))\{}
\NormalTok{  titik }\OtherTok{=} \FunctionTok{c}\NormalTok{(df}\SpecialCharTok{$}\NormalTok{x[i],df}\SpecialCharTok{$}\NormalTok{y[i])}
\NormalTok{  df}\SpecialCharTok{$}\NormalTok{jarak\_sentroid1[i] }\OtherTok{=} \FunctionTok{jarak}\NormalTok{(titik,sentroid\_1)}
\NormalTok{  df}\SpecialCharTok{$}\NormalTok{jarak\_sentroid2[i] }\OtherTok{=} \FunctionTok{jarak}\NormalTok{(titik,sentroid\_2)}
\NormalTok{\}}

\NormalTok{df }\OtherTok{=} 
\NormalTok{  df }\SpecialCharTok{\%\textgreater{}\%} 
  \FunctionTok{mutate}\NormalTok{(}\AttributeTok{membership =} \FunctionTok{ifelse}\NormalTok{(jarak\_sentroid1 }\SpecialCharTok{\textless{}}\NormalTok{ jarak\_sentroid2,}\DecValTok{1}\NormalTok{,}\DecValTok{2}\NormalTok{))}
\end{Highlighting}
\end{Shaded}

\newpage

\hypertarget{soal-ii}{%
\subsection{Soal II}\label{soal-ii}}

Diberikan \emph{confusion matrix} sebagai berikut:

\begin{longtable}[]{@{}
  >{\centering\arraybackslash}p{(\columnwidth - 14\tabcolsep) * \real{0.1184}}
  >{\centering\arraybackslash}p{(\columnwidth - 14\tabcolsep) * \real{0.1974}}
  >{\centering\arraybackslash}p{(\columnwidth - 14\tabcolsep) * \real{0.1447}}
  >{\centering\arraybackslash}p{(\columnwidth - 14\tabcolsep) * \real{0.1184}}
  >{\centering\arraybackslash}p{(\columnwidth - 14\tabcolsep) * \real{0.0921}}
  >{\centering\arraybackslash}p{(\columnwidth - 14\tabcolsep) * \real{0.1316}}
  >{\centering\arraybackslash}p{(\columnwidth - 14\tabcolsep) * \real{0.1053}}
  >{\centering\arraybackslash}p{(\columnwidth - 14\tabcolsep) * \real{0.0921}}@{}}
\caption{Data Soal II}\tabularnewline
\toprule
\begin{minipage}[b]{\linewidth}\centering
cluster
\end{minipage} & \begin{minipage}[b]{\linewidth}\centering
entertainment
\end{minipage} & \begin{minipage}[b]{\linewidth}\centering
financial
\end{minipage} & \begin{minipage}[b]{\linewidth}\centering
foreign
\end{minipage} & \begin{minipage}[b]{\linewidth}\centering
metro
\end{minipage} & \begin{minipage}[b]{\linewidth}\centering
national
\end{minipage} & \begin{minipage}[b]{\linewidth}\centering
sports
\end{minipage} & \begin{minipage}[b]{\linewidth}\centering
Total
\end{minipage} \\
\midrule
\endfirsthead
\toprule
\begin{minipage}[b]{\linewidth}\centering
cluster
\end{minipage} & \begin{minipage}[b]{\linewidth}\centering
entertainment
\end{minipage} & \begin{minipage}[b]{\linewidth}\centering
financial
\end{minipage} & \begin{minipage}[b]{\linewidth}\centering
foreign
\end{minipage} & \begin{minipage}[b]{\linewidth}\centering
metro
\end{minipage} & \begin{minipage}[b]{\linewidth}\centering
national
\end{minipage} & \begin{minipage}[b]{\linewidth}\centering
sports
\end{minipage} & \begin{minipage}[b]{\linewidth}\centering
Total
\end{minipage} \\
\midrule
\endhead
\#1 & 1 & 1 & 0 & 11 & 4 & 676 & 693 \\
\#2 & 27 & 89 & 333 & 827 & 253 & 33 & 1562 \\
\#3 & 326 & 465 & 8 & 105 & 16 & 29 & 949 \\
Total & 354 & 555 & 341 & 943 & 273 & 738 & 3204 \\
\bottomrule
\end{longtable}

\hypertarget{pertanyaan-1}{%
\subsubsection{Pertanyaan}\label{pertanyaan-1}}

Hitung nilai \emph{entropy} dan \emph{purity} untuk matriks tersebut!
Berikan analisis untuk hasil yang didapat!

\hypertarget{pembahasan-1}{%
\subsubsection{Pembahasan}\label{pembahasan-1}}

Entropi untuk masing-masing cluster dihitung sebagai berikut:

Sedangkan untuk \emph{purity} dihitung dengan cara:

\[
\begin{align*}
\text{Purity 1} & = \frac{676}{693} &= 0.975 \\
\text{Purity 2} & = \frac{827}{1562} &= 0.529 \\
\text{Purity 3} & = \frac{465}{949} &= 0.490 \\
\end{align*}
\]

\emph{Total entropy} dihitung sebagai berikut:

\[
\text{Total entropy} = \frac{693 \times 0.200 + 1562 \times 1.841 + 949 \times 0.490}{3204} = 0.614
\]

\emph{Total purity} dihitung sebagai berikut:

\[
\text{Total purity} = \frac{693 \times 0.975 + 1562 \times 0.529 + 949 \times 1.696}{3204} = 1.443
\]

\end{document}
