% Options for packages loaded elsewhere
\PassOptionsToPackage{unicode}{hyperref}
\PassOptionsToPackage{hyphens}{url}
\PassOptionsToPackage{dvipsnames,svgnames,x11names}{xcolor}
%
\documentclass[
  12pt,
]{article}
\usepackage{amsmath,amssymb}
\usepackage{lmodern}
\usepackage{setspace}
\usepackage{iftex}
\ifPDFTeX
  \usepackage[T1]{fontenc}
  \usepackage[utf8]{inputenc}
  \usepackage{textcomp} % provide euro and other symbols
\else % if luatex or xetex
  \usepackage{unicode-math}
  \defaultfontfeatures{Scale=MatchLowercase}
  \defaultfontfeatures[\rmfamily]{Ligatures=TeX,Scale=1}
\fi
% Use upquote if available, for straight quotes in verbatim environments
\IfFileExists{upquote.sty}{\usepackage{upquote}}{}
\IfFileExists{microtype.sty}{% use microtype if available
  \usepackage[]{microtype}
  \UseMicrotypeSet[protrusion]{basicmath} % disable protrusion for tt fonts
}{}
\makeatletter
\@ifundefined{KOMAClassName}{% if non-KOMA class
  \IfFileExists{parskip.sty}{%
    \usepackage{parskip}
  }{% else
    \setlength{\parindent}{0pt}
    \setlength{\parskip}{6pt plus 2pt minus 1pt}}
}{% if KOMA class
  \KOMAoptions{parskip=half}}
\makeatother
\usepackage{xcolor}
\usepackage[margin=1in]{geometry}
\usepackage{graphicx}
\makeatletter
\def\maxwidth{\ifdim\Gin@nat@width>\linewidth\linewidth\else\Gin@nat@width\fi}
\def\maxheight{\ifdim\Gin@nat@height>\textheight\textheight\else\Gin@nat@height\fi}
\makeatother
% Scale images if necessary, so that they will not overflow the page
% margins by default, and it is still possible to overwrite the defaults
% using explicit options in \includegraphics[width, height, ...]{}
\setkeys{Gin}{width=\maxwidth,height=\maxheight,keepaspectratio}
% Set default figure placement to htbp
\makeatletter
\def\fps@figure{htbp}
\makeatother
\setlength{\emergencystretch}{3em} % prevent overfull lines
\providecommand{\tightlist}{%
  \setlength{\itemsep}{0pt}\setlength{\parskip}{0pt}}
\setcounter{secnumdepth}{-\maxdimen} % remove section numbering
\newlength{\cslhangindent}
\setlength{\cslhangindent}{1.5em}
\newlength{\csllabelwidth}
\setlength{\csllabelwidth}{3em}
\newlength{\cslentryspacingunit} % times entry-spacing
\setlength{\cslentryspacingunit}{\parskip}
\newenvironment{CSLReferences}[2] % #1 hanging-ident, #2 entry spacing
 {% don't indent paragraphs
  \setlength{\parindent}{0pt}
  % turn on hanging indent if param 1 is 1
  \ifodd #1
  \let\oldpar\par
  \def\par{\hangindent=\cslhangindent\oldpar}
  \fi
  % set entry spacing
  \setlength{\parskip}{#2\cslentryspacingunit}
 }%
 {}
\usepackage{calc}
\newcommand{\CSLBlock}[1]{#1\hfill\break}
\newcommand{\CSLLeftMargin}[1]{\parbox[t]{\csllabelwidth}{#1}}
\newcommand{\CSLRightInline}[1]{\parbox[t]{\linewidth - \csllabelwidth}{#1}\break}
\newcommand{\CSLIndent}[1]{\hspace{\cslhangindent}#1}
\usepackage{fancyhdr}
\pagestyle{fancy}
\fancyfoot[CO,CE]{20921004@mahasiswa.itb.ac.id}
\fancyfoot[LE,RO]{hal \thepage}
\ifLuaTeX
  \usepackage{selnolig}  % disable illegal ligatures
\fi
\IfFileExists{bookmark.sty}{\usepackage{bookmark}}{\usepackage{hyperref}}
\IfFileExists{xurl.sty}{\usepackage{xurl}}{} % add URL line breaks if available
\urlstyle{same} % disable monospaced font for URLs
\hypersetup{
  pdftitle={Laporan Akhir Penelitian Mandiri dalam Sains Komputasi IV},
  colorlinks=true,
  linkcolor={Maroon},
  filecolor={Maroon},
  citecolor={Blue},
  urlcolor={Blue},
  pdfcreator={LaTeX via pandoc}}

\title{Laporan Akhir Penelitian Mandiri dalam Sains Komputasi IV}
\usepackage{etoolbox}
\makeatletter
\providecommand{\subtitle}[1]{% add subtitle to \maketitle
  \apptocmd{\@title}{\par {\large #1 \par}}{}{}
}
\makeatother
\subtitle{Model Optimisasi}
\author{Mohammad Rizka Fadhli\\
Program Studi Magister Sains Komputasi\\
Fakultas Matematika dan Ilmu Pengetahuan Alam\\
Institut Teknologi Bandung\\
\href{mailto:20921004@mahasiswa.itb.ac.id}{\nolinkurl{20921004@mahasiswa.itb.ac.id}}}
\date{19 August 2022}

\begin{document}
\maketitle

\setstretch{1.75}
\newpage
\tableofcontents
\newpage
\listoffigures
\newpage
\listoftables

\newpage

\hypertarget{latar-belakang}{%
\section{LATAR BELAKANG}\label{latar-belakang}}

Semenjak diperkenalkan pertama kali pada tahun 1982, \emph{Supply Chain
Management} (SCM) memegang peranan penting dalam manufaktur sebagai
suatu sistem produksi terintegrasi {[}1{]}. Di dalam SCM, bahan baku
dibeli perusahaan dari berbagai \emph{supplier}, dibuat ke dalam suatu
produk yang kemudian akan dijual ke pelanggan melalui berbagai
\emph{channel} distribusi.

Dalam mengarungi kompetisi, perusahaan perlu memahami dua faktor kunci,
yakni \emph{cost reduction} dan \emph{product quality} {[}2{]}. Kedua
faktor ini sangat bergantung pada pemilihan \emph{supplier} yang tepat.
Sehingga proses \emph{supplier selection} menjadi proses yang krusial
dalam setiap perusahaan.

Dalam prakteknya, perusahaan bisa menggunakan dua strategi terkait
\emph{supplier selection}, yakni: \emph{single sourcing} dan
\emph{multiple sourcing}. \emph{Single sourcing} berarti perusahaan
hanya membeli bahan baku dari \emph{supplier} tunggal. Sedangkan
\emph{multiple sourcing} berarti perusahaan bisa membeli bahan baku dari
beberapa \emph{supplier}. Strategi \emph{single sourcing} bisa menaikkan
level risiko dari perusahaan sedangkan strategi \emph{multiple sourcing}
menyebabkan \emph{initial cost} dan \emph{ongoing cost} yang lebih besar
{[}3{]}. Bagi perusahaan yang menerapkan strategi \emph{multiple
sourcing}, banyak faktor yang akan membuat kompleks pengambilan
keputusan. Misalnya harga, perjanjian transaksi, kualitas, kuantitas,
jarak dan biaya pengantaran {[}2{]}.

PT. NFI adalah salah satu perusahaan manufaktur di Indonesia yang
memproduksi 130 jenis minuman. Salah satu bahan baku yang paling sering
digunakan untuk semua produk minuman tersebut adalah gula. Masing-masing
produk minuman tersebut bisa dibagi menjadi dua kelompok, yakni:

\begin{enumerate}
\def\labelenumi{\arabic{enumi}.}
\tightlist
\item
  Minuman yang hanya bisa diproduksi oleh satu jenis bahan baku gula.
\item
  Minuman yang bisa diproduksi menggunakan dua atau lebih jenis bahan
  baku gula.
\end{enumerate}

Untuk pemenuhan bahan baku gula, NFI menggunakan prinsip \emph{multiple
sourcing} dengan perjanjian untuk memasoknya dari enam buah
\emph{supplier}. Spesifikasi bahan baku gula dan harga perton
berbeda-beda antar \emph{supplier}.

Pada penelitian ini, ada tiga masalah utama yang hendak diselesaikan,
yakni:

\begin{itemize}
\tightlist
\item
  Memilih \emph{supplier} bahan baku.
\item
  Menentukan banyaknya bahan baku yang harus dibeli dari suatu
  \emph{supplier}.
\item
  Menentukan bahan baku mana yang harus digunakan untuk memproduksi
  setiap produk.
\end{itemize}

dengan tujuan total biaya pembelian seminim mungkin tetapi memenuhi
kebutuhan yang ada pada periode tertentu. Luaran dari penelitian ini
adalah suatu model optimisasi yang bisa menyelesaikan permasalahan di
atas.

Laporan akhir penelitian mandiri dalam sains komputasi IV ini adalah:
model optimisasi yang telah disempurnakan untuk penelitian berjudul
\emph{Optimization Method for Supplier Selection, Order Allocation, and
Incorporating Raw-Material Characteristic: Case Study Beverages
Manufacture}.

\hypertarget{model-optimisasi}{%
\section{MODEL OPTIMISASI}\label{model-optimisasi}}

Masalah \emph{supplier selection}, \emph{order allocation}, dan
pemasangan bahan baku dengan produk adalah masalah dengan satu kriteria,
yaitu total harga pengadaan (pembelian). Selain itu, ada satu variabel
keputusan lain yang hendak dicari, yakni bagaimana distribusi pengiriman
dari tiap \emph{supplier} per minggu. Oleh karena itu, masalah krusial
pertama dari penyelesaian masalah ini adalah menurunkan masalah
optimisasi yang tepat yang dapat menjadi model dari masalah ini.

Berdasarkan informasi-informasi yang telah didapatkan dari
\texttt{Penelitian\ Mandiri\ III}, berikut adalah model optimisasi dari
permasalahan ini.

\hypertarget{parameter-yang-diketahui}{%
\subsection{Parameter yang Diketahui}\label{parameter-yang-diketahui}}

Notasikan:

\begin{itemize}
\tightlist
\item
  \(M = {1,2,3,4,5,6}\) sebagai himpunan semua minggu.
\item
  \(P = P_1 \bigcup P_2 \bigcup P_3 \bigcup P_4 \bigcup P_5 \bigcup P_6\)
  sebagai himpunan produk yang diproduksi per minggu.
\item
  \(G = \{1,2,3,4,5,6\}\) sebagai himpunan bahan baku.
\item
  \(D\) sebagai kebutuhan bahan baku (\emph{demand}) di bulan
  perencanaan, yaitu: \emph{week} 3 - 6.
\item
  \(maxcap\) sebagai kapasitas gudang bahan baku.
\item
  \(\forall i \in P_j, \space g_{ijk}\) adalah kebutuhan bahan baku
  \(k\) (dalam ton) dari produk \(i\) pada \emph{week} \(j\).
\item
  \(\forall k \in G, Pr_k\) adalah total proporsi portofolio bahan baku
  yang ditetapkan dalam setahun (dalam ton).
\item
  \(\forall k \in G, \space c_k\) adalah harga bahan baku \(k\) per ton.
\item
  \(\forall k \in G, \space o_k\) adalah \emph{minimum order quantity}
  dari bahan baku \(k\).
\item
  \(\forall k \in G, \space \hat{d}_{2k}\) adalah total bahan baku \(k\)
  yang dibutuhkan pada \emph{week} 2.
\item
  \(\forall k \in G, \space Z_{1k}\) adalah stok level bahan baku \(k\)
  di gudang pada akhir \emph{week} 1.
\end{itemize}

\hypertarget{variabel-keputusan}{%
\subsection{Variabel Keputusan}\label{variabel-keputusan}}

\hypertarget{variabel-i}{%
\subsubsection*{Variabel I}\label{variabel-i}}

Definisikan \(\forall k \in G, \space x_k\) adalah banyaknya bahan baku
\(k\) yang dibeli.

Berdasarkan informasi sebelumnya, kita ketahui bahwa \(x_k\) bernilai
bulat positif dan harus lebih besar atau sama dengan nilai \(o_k\).
Kemudian tidak ada kewajiban untuk membeli bahan baku dari seluruh
\emph{supplier}.

Maka kita bisa tuliskan: \(x_k = 0\) atau \(o_k \leq x_k \leq maxcap\).
Untuk menghindari nilai diskontinu dari \(x_k\) ini, definisikan:

\[y_k = \left\{\begin{matrix}
1, & x_k = 0 \\ 0, & o_k \leq x_k \leq maxcap
\end{matrix}\right.\]

\(\forall j \in M \setminus \{1,6\}, \forall i \in P_j, \forall k \in G\),

\hypertarget{variabel-ii}{%
\subsubsection*{Variabel II}\label{variabel-ii}}

Definisikan: \(\hat{x}_{jk}\) sebagai banyaknya pengiriman bahan baku
jenis \(k\) di awal \emph{week} \(j\).

\[a_{ijk} = \left\{\begin{matrix}
1, & \text{produk ke } i \text{ di week } j \text{ menggunakan BB } k \\ 
0, & \text{lainnya}
\end{matrix}\right.\]

\hypertarget{variabel-iii}{%
\subsubsection*{Variabel III}\label{variabel-iii}}

Definisikan: \(b_{ijk}\) sebagai proporsi penggunaan bahan baku \(k\)
dari seluruh kebutuhan bahan baku untuk produk \(i\) di \emph{week}
\(j\), \(\forall j \in M \setminus \{ 1 \}, \forall k \in G\).

\hypertarget{variabel-iv}{%
\subsubsection*{Variabel IV}\label{variabel-iv}}

Definisikan: \(z_{jk}\) sebagai stok level bahan baku \(k\) di akhir
\emph{week} \(j\).

\hypertarget{kendala-optimisasi}{%
\subsection{Kendala Optimisasi}\label{kendala-optimisasi}}

\hypertarget{kendala-i}{%
\subsubsection*{Kendala I}\label{kendala-i}}

Kendala I adalah penghubung yang benar antara variabel keputusan biner,
integer, atau kontinu yang berkaitan:

\[\begin{matrix}
\forall k \in G, &  \\
 & x_k \leq Dy_k \\
 & x_k \geq \epsilon y_k \\
\forall j \in M \setminus \{ 1,2 \}, \forall i \in P_j, \forall k \in G, & \\
 & b_{ijk} \leq a_{ijk} \\
 & b_{ijk} \geq \mu a_{ijk}
\end{matrix}\]

untuk suatu nilai \(\mu\) yang kecil.

\hypertarget{kendala-ii}{%
\subsubsection*{Kendala II}\label{kendala-ii}}

Kendala II dibuat agar total bahan baku yang dipesan tidak kurang dari
total \emph{demand} di bulan perencanaan.

\[\sum_{k \in G} x_k \geq D\]

\hypertarget{kendala-iii}{%
\subsubsection*{Kendala III}\label{kendala-iii}}

Kendala III mengatur hubungan antara total pembelian bahan baku dan
pengiriman setiap minggu.

\[\begin{matrix}
\forall k \in G, & \\
 & x_k = \sum_{j \in \hat{M}} \hat{x}_{jk}
\end{matrix}\]

\hypertarget{kendala-iv}{%
\subsubsection*{Kendala IV}\label{kendala-iv}}

Kendala IV berfungsi untuk menjaga komposisi bahan baku yang diinginkan.

\[\begin{matrix}
\forall j \in M \setminus \{ 1,2 \}, \space \forall i \in \hat{P}_j, & \\
 & \sum_{k \in G} a_{ijk} \geq 2 \\
 & \sum_{k \in G} b_{ijk} = 1 \\
\forall j \in M \setminus \{ 1,2 \}, \space \forall i \in \dot{P}_j, & \\ 
 & \sum_{k \in G} a_{ijk} = 1 \\
 & \sum_{k \in G} b_{ijk} = 1
\end{matrix}\]

\hypertarget{kendala-v}{%
\subsubsection*{Kendala V}\label{kendala-v}}

Kendala V berfungsi untuk menjaga stok level sesaat setelah pengiriman
bahan baku agar tidak melebihi kapasitas gudang.

\[\begin{matrix}
 & \sum_{k \in G} (Z_{1k} + \hat{x}_{1k} - \hat{d}_{2k} + z_{jk}) = maxcap \\
\forall j \in M \setminus \{ 1,2 \} & \\
 & \sum_{k \in G} (z_{(j-1)k} + \hat{x}_{(j-1)k}) - \sum_{i \in P_j} b_{ijk} g_{ijk} + z_{jk} = maxcap
\end{matrix}\]

\hypertarget{kendala-vi}{%
\subsubsection*{Kendala VI}\label{kendala-vi}}

Kendala VI menjaga agar pembelian bahan baku tidak melebihi proporsi
portofolio yang sudah ditetapkan dalam setahun.

\[\sum_{k \in G} x_k \leq Pr_k\]

\hypertarget{fungsi-objektif}{%
\subsection{Fungsi Objektif}\label{fungsi-objektif}}

Permasalahan yang dihadapi adalah pemilihan \emph{supplier} dan bahan
baku sebagai berikut:

\[\min \sum_{k \in G} c_k x_k\]

\[\begin{matrix}
\text{terhadap kendala I sampai VI dan } \\
x_k = 0 \text{ atau } o_k \leq x_k \leq maxcap, \space x_k \in \mathbb{Z}^+ \\
y_k \in \{ 0,1 \}, \hat{x}_{jk} \geq 0, a_{ijk} \in \{ 0,1 \} \\
0 \leq b_{ijk} \leq 1 \\
0 \leq z_{jk} \leq maxcap \\
\end{matrix}\]

\newpage

\hypertarget{referensi}{%
\section*{REFERENSI}\label{referensi}}
\addcontentsline{toc}{section}{REFERENSI}

\hypertarget{refs}{}
\begin{CSLReferences}{0}{0}
\leavevmode\vadjust pre{\hypertarget{ref-webber}{}}%
1. Oliver RK, Webber MD. Supply-chain management: Logistics catches up
with strategy. Outlook; 1982.

\leavevmode\vadjust pre{\hypertarget{ref-masood}{}}%
2. Rabieh M, Soukhakian MA, Shirazi ANM. Two models of inventory control
with supplier selection in case of multiple sourcing: A case of isfahan
steel company. Springerlink.com; 2016.

\leavevmode\vadjust pre{\hypertarget{ref-sourcing}{}}%
3. Costantino N, Pellegrino R. Choosing between single and multiple
sourcing based on supplier default risk: A real options approach.
Journal of Purchasing; Supply Management; 2010.

\end{CSLReferences}

\end{document}
